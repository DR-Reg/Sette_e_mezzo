\documentclass[12pt]{article}
\usepackage[margin=0.25in]{geometry}
\usepackage{pgfplots}
\usepackage{amsmath}
\pgfplotsset{width=10cm,compat=1.9}

\usepgfplotslibrary{external}
\tikzexternalize

\title{Sette e Mezzo}
\author{David Raibaut}
\date{11 Sep 2023}

\begin{document}
\maketitle
\newpage

\section{Introduction}
\subsection{Game Mechanics}
\textit{Sette e Mezzo} in Italian, \textit{Siete y Medio} in Spanish or 'seven and a half'
in English is a comparing card game which can is usually played Italian playing cards or Spanish
suits, however it can be played with any regular 40-card deck. The value of the cards ace, 2, 3, 4, 5, 6, 7 is assigned
its numeric value, whilst face cards are assigned the value of a half (8s and 9s are taken out).
One player is the dealer with all the others betting against them.
\\
In each round, a player sets some sum of money forth their 'bet' for this round. They must bet before receiving their first card.
The first card is always dealt face down. The player can then decide to stand (end their turn) or hit (receive another card).
They can do this indefinitely until they stand or go bust (i.e. go over the maximum score, 7.5). You can increase or decrease your bet
after each turn, until you stand. Once everyone has stood or gone bust, the dealer begins playing with the same rules.
\\
Once the dealer played, everyone has to show their hand. If their score is larger than the dealer's, they get paid the amount of money they bet.
If they have achieved 7.5, they gain twice this. If they tie or lose to the dealer, they have to pay their bet.
\subsection{Motivation}
TODO
\section{Pot odds}

\section{Model}
\subsection{Markov Chain}
\subsection{Computer Simulations}
\subsection{Stochastic Model} % i.e. probability
\section{Splitting}

\section{Assumptions,Limitations,Future work}

\section{Conclusion}

% REFERENCES!!! - bibtex

\end{document}